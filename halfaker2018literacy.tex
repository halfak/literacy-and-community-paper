\documentclass{sigchi-ext}
% Please be sure that you have the dependencies (i.e., additional
% LaTeX packages) to compile this example.
\usepackage[T1]{fontenc}
\usepackage{textcomp}
\usepackage[scaled=.92]{helvet} % for proper fonts
\usepackage{graphicx} % for EPS use the graphics package instead
\usepackage{balance}  % for useful for balancing the last columns
\usepackage{booktabs} % for pretty table rules
\usepackage{ccicons}  % for Creative Commons citation icons
\usepackage{ragged2e} % for tighter hyphenation

% Some optional stuff you might like/need.
% \usepackage{marginnote}
% \usepackage[shortlabels]{enumitem}
% \usepackage{paralist}
% \usepackage[utf8]{inputenc} % for a UTF8 editor only

%% EXAMPLE BEGIN -- HOW TO OVERRIDE THE DEFAULT COPYRIGHT STRIP --
\copyrightinfo{This article is licensed under the Creative Commons \\
Attribution 4.0 International license (CC BY 4.0). You are free to share\\
and adapt this work, provided you attribute the authors and leave this \\ copyright notice intact.
 {\emph{CSCW'18}}, November 3-7, 2018, Jersey City, NJ, USA. \\
 https://doi.org/XXXXX/XXXXX}

% Paper metadata (use plain text, for PDF inclusion and later
% re-using, if desired).  Use \emtpyauthor when submitting for review
% so you remain anonymous.
\def\plaintitle{Community Literacy of Machine Learning: Engineering practical support for participatory design and auditing}
\def\plainkeywords{Algorithm; Fairness; Transparency; Wikipedia; Collaboration; Participatory design; Auditing}
\def\plaingeneralterms{Algorithm, Fairness, Transparency, Wikipedia, Collaboration, Participatory, Auditing}
\title{Community Literacy of Machine Learning: Engineering practical support for participatory design and auditing}


\def\plainauthor{Aaron Halfaker and Stuart Geiger}
\def\emptyauthor{}

\numberofauthors{2}
% Notice how author names are alternately typesetted to appear ordered
% in 2-column format; i.e., the first 4 autors on the first column and
% the other 4 auhors on the second column. Actually, it's up to you to
% strictly adhere to this author notation.
\author{%
  \alignauthor{%
    \textbf{Aaron Halfaker}\\
    \affaddr{Wikimedia Research} \\
    \affaddr{San Francisco, CA, USA} \\
    \email{ahalfaker@wikimedia.org} }\alignauthor{%
    \textbf{Stuart Geiger}\\
    \affaddr{Univ. of California, Berkely}\\
    \affaddr{Berkeley Inst. of Data Science}\\
    \affaddr{Berkeley, CA, USA}\\
    \email{sgeiger@gmail.com} } }


% Make sure hyperref comes last of your loaded packages, to give it a
% fighting chance of not being over-written, since its job is to
% redefine many LaTeX commands.
\definecolor{linkColor}{RGB}{6,125,233}
\hypersetup{%
  pdftitle={\plaintitle},
%  pdfauthor={\plainauthor},
  pdfauthor={\emptyauthor},
  pdfkeywords={\plainkeywords},
  bookmarksnumbered,
  pdfstartview={FitH},
  colorlinks,
  citecolor=black,
  filecolor=black,
  linkcolor=black,
  urlcolor=linkColor,
  breaklinks=true,
}

% \reversemarginpar%

\begin{document}

%% For the camera ready, use the commands provided by the ACM in the Permission Release Form.
%\CopyrightYear{2007}
%\setcopyright{rightsretained}
%\conferenceinfo{WOODSTOCK}{'97 El Paso, Texas USA}
%\isbn{0-12345-67-8/90/01}
%\doi{http://dx.doi.org/10.1145/2858036.2858119}
%% Then override the default copyright message with the \acmcopyright command.
%\copyrightinfo{\acmcopyright}

\maketitle

% Uncomment to disable hyphenation (not recommended)
% https://twitter.com/anjirokhan/status/546046683331973120
\RaggedRight{}

% Do not change the page size or page settings.
\begin{abstract}
ORES is cool.  People do fun stuff around it.  Thinking
about transparency in terms of crowd literacy is interesting.
We see evidence of crowd literacy growing around ORES.
Crowd literacy allows groups of people to exert power
over governing algorithms.
\end{abstract}

\keywords{\plainkeywords}

\category{G.4}{MATHEMATICAL SOFTWARE}{Algorithm design and analysis}
\category{H.4.2}{Types of Systems}{Decision support (e.g., MIS)}
\category{H.5.3}{Group and Organization Interfaces}{Collaborative computing}

\section{Introduction}
I am a paper

\section{Literacy of social algorithms}
It's hard.

\section{CSCW and group knowledges}
Communities of practice!  Distributed cognition!  There are ways that groups \emph{think}.\cite{crawford2016algorithm}

\section{ORES: Participatory machine learning probe}
ORES is a system.  It's pretty transparent.  We deployed it in Wikipedia.  People are doing some very CSCWy things around it and showing deep literacy.

\section{Discussion}
We should think of ML literacy differently.  Collaborative/participatory processes allow for natural, social processes to support literacy.
We should build a proper auditing system to support this stuff.
Power struggles are real.  This isn't the holy grail.  But it surfaces some cool stuff.

\balance{}

\bibliographystyle{SIGCHI-Reference-Format}
\bibliography{refs}

\end{document}

%%% Local Variables:
%%% mode: latex
%%% TeX-master: t
%%% End:
